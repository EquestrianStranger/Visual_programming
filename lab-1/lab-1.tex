\documentclass{article}
\usepackage[utf8x]{inputenc}
\usepackage[T2A]{fontenc}
\usepackage[russian]{babel}

\usepackage{xcolor}
\usepackage{listings}
\usepackage{caption}
\usepackage[left=2cm,right=2cm,top=1cm,bottom=3cm]{geometry}

\definecolor{backgroundColour}{rgb}{0.97,0.92,0.84}
\definecolor{maroon}{rgb}{0.5,0,0}

\DeclareCaptionFont{white}{\color{white}}
\DeclareCaptionFormat{listing}{\colorbox{maroon}{\parbox{\dimexpr\textwidth-2\fboxsep\relax}{#1#2#3}}}
\captionsetup[lstlisting]{format=listing,labelfont=white,textfont=white}

\lstset{
language=C++,
backgroundcolor=\color{backgroundColour},
title=\lstname,
basicstyle=\ttfamily\footnotesize,
keywordstyle=\color{blue},
stringstyle=\color{red},
commentstyle=\color{green},
numberstyle=\tiny,
numbers=left,
stepnumber=1,
numbersep=5pt,
showspaces=false,
showstringspaces=false,
showtabs=false,
frame=single,
tabsize=2,
captionpos=t,
breaklines=true,
breakatwhitespace=false
}


\begin{document}


\title{Стиль кода}
\author{Автор: Кочкин Иван Денисович, ИА-032,\\ email: vaniakochkin24@gmail.com,\\ github: EquestrianStranger}
\date{Февраль 2022}

\maketitle

\section{Введение}
\textbf{Стиль кода} - это набор правил для написания проекта. Единый стиль упрощает чтение, редактирование и написание кода.

\section{C}
Используется версия С17 \cite{C}.\vspace{5mm}

\textbf{Инициализация переменных.} Ставятся пробелы до и после знака присваивания <<=>>.
\begin{lstlisting}[caption=Инициализация переменных.]
int a = 1;
char n;
\end{lstlisting}

\textbf{Ввод и вывод данных.} Пробел ставится после запятой.
\begin{lstlisting}[caption=Ввод и вывод данных.]
scanf("%c", &n);
printf("Number - %d", a);
\end{lstlisting}

\textbf{Условный оператор if.} Ставятся пробелы после <<if>> и <<else>>, до и после знаков опираций. В виде отступов используется табуляция (2 пробела).
\begin{lstlisting}[caption=Условный оператор if.]
if (a < 10){
  a = a + 1;
}
else{
  a = a - 2;
}
\end{lstlisting}

\textbf{Цикл for.} Использование стиля аналогично условному опреатору if.
\begin{lstlisting}[caption=Цикл for.]
for (int i = 2; i < n; i++){
    y = i * y;
    x = i * i;
}
\end{lstlisting}

\textbf{Реализация функции}, например, swapInt.
\begin{lstlisting}[caption=Реализация функции.]
void swapInt(int* a, int* b){
  *a ^= *b;
  *b ^= *a;
  *a ^= *b;
}
\end{lstlisting}

\section{C++}
Используется версия С++20 \cite{C++}.\vspace{5mm}

Применяются стили написания, аналогично языку C. Поэтому следует описать ввод и вывод данных, а также классы, которые не используются в языке C.\par
\textbf{Ввод и вывод данных.} Ставятся пробелы до и после операторов ввода и вывода.
\begin{lstlisting}[caption=Ввод и вывод данных.]
std::cout << "Choose a number: ";
std::cin >> a;
std::cout << "Your number is" << a << std::endl;
\end{lstlisting}

\textbf{Классы.}, например, ManufactFirm.
\begin{lstlisting}[caption=Классы.]
class ManufactFirm{
  private:
  std::string name;
  std::string country;
  std::string equipmentType;

  public:
  ManufactFirm(std::string m_nm, std::string m_cntr, std::string m_eqtp){
    name = m_nm;
    country = m_cntr;
    equipmentType = m_eqtp;
  }

  void print(){
    std::cout << country << " " << equipmentType << "\n";
  }

  std::string get_country(){
    return country;
  }

  void set_name(std::string m_nm){
    name = m_nm;
  }
};
\end{lstlisting}

\begin{thebibliography}{}
\bibitem{C}
ISO/IEC 9899 Programming languages~--- C.
\bibitem{C++}
ISO/IEC 14882 Programming languages~--- C++.
\end{thebibliography}

\end{document}